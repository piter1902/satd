\documentclass[10pt,a4paper]{article}
\usepackage[utf8]{inputenc}
\usepackage[spanish]{babel}
\usepackage{amsmath}
\usepackage{amsfonts}
\usepackage{amssymb}
\usepackage{graphicx}
\usepackage[left=2cm,right=2cm,top=2cm,bottom=2cm]{geometry}
\usepackage[hidelinks]{hyperref}
\usepackage{listings}

\lstset { frame = single }

\begin{document}

\begin{titlepage}
\title{\textbf{
	{\Huge Práctica 3: Integración de \emph{KNIME} y \emph{WEKA}}\\
	{\Large Sistemas de Ayuda a la Toma de Decisiones}
}}
\author{
	Pedro Allué Tamargo (758267)
	\and
	Juan José Tambo Tambo (755742)
	\and
	Jesús Villacampa Sagaste (755739)
}
\date{\today}
\clearpage\maketitle
\thispagestyle{empty}
\end{titlepage}

\tableofcontents

\newpage
\section{Ejercicio 1}

Se ha creado el \emph{workflow} ilustrado en la Figura \textbf{numFigura} para trabajar con los datos del conjunto de datos \emph{yellow-small.data}.\\
Se ha utilizado un nodo \emph{Rule Engine} para crear una nueva columna \emph{``class''}. El contenido de este nodo son las siguientes reglas:

\begin{lstlisting}
$Color$ MATCHES "YELLOW" AND $Size$ MATCHES "SMALL" => "inflated"
TRUE => "not inflated"
\end{lstlisting}

Tras este nodo se ha utilizado un nodo \emph{String manipulation} para concatenar los valores de las columnas \emph{class} e \emph{inflated (true/false)} utilizando la expresión:

\begin{lstlisting}
string($class$ + " is " + $Inflated (True/False)$)
\end{lstlisting}

{\Huge \textbf{Meter figura aqui}}

\section{Ejercicio 2}

Se va a proceder a analizar un conjunto de datos que describe el número de visitantes de un sitio web en los meses de junio/julio de 2010 (fichero \emph{website1.txt}).\\

Para calcular los parámetros de media, desviación típica, Kurtosis se ha utilizado el \emph{workflow} mostrado en la Figura \textbf{numFigura}.\\
La \emph{Kurtosis} es una medida estadística que muestra la forma de una distribución de probabilidad. Una \emph{Kurtosis} grande implica una mayor concentración de valores de la variables o muy cerca de la media de la distribución (pico) o muy lejos de ella (colas de la distribución), al mismo tiempo que existe una menor frecuencia de valores intermedios.\\
\textbf{¿Cómo ilustramos esto en este conjunto de datos?}\\

Para entrenar la red Bayesiana (Figura \textbf{numFigura}) se deben preparar los datos. Para ello se debe crear una nueva columna \emph{isWeekend} para ilustrar si es fin de semana o no. Utilizando el nodo \emph{Rule engine} se utilizarán las siguientes reglas:

\begin{lstlisting}
$weekday$ MATCHES "Sat" => "Yes"
$weekday$ MATCHES "Sun" => "Yes"
TRUE => "No"
\end{lstlisting}

Se ha utilizado un nodo \emph{Column filter} para eliminar la columna \emph{weekday} ya que la red Bayesiana presenta un mejor rendimiento si conoce este valor ya que si se entrena con esta variable reconoce la regla de creación de la columna \emph{isWeekend}.\\
Para llegar a esta conclusión se han probado las distintas combinaciones de columnas utilizando el \emph{Column Filter}.\\
\textbf{¿Explica más pruebas realizadas?}\\

Para dibujar la curva \emph{ROC} se ha utilizado un nodo \emph{ROC Cuve (local)} a la salida del nodo \emph{Naive Bayes Predictor}. Se puede observar en la Figura \textbf{numFigura} que la forma de la gráfica... \textbf{(?)}.\\


{\Huge \textbf{Meter figura aqui}}

{\Huge \textbf{Meter gráfica aqui}}


\section{Ejercicio 3}

Para la evaluación de los distintos conjuntos de datos se han creado 2 \emph{workflows}. Uno de ellos (Figura \textbf{numFigura}) utiliza las herramientas de \emph{KNIME} para evaluar los datos. El otro (Figura \textbf{numFigura}) utiliza las herramientas de \emph{WEKA} para evaluar los datos.\\

Para el conjunto de datos \emph{wine} se han entrenado las distintas herramientas con el 80\%, 50\% y 30\%.\\
Para el conjunto de datos \emph{iris} se han entrenado las distintas herramientas con el 80\%, 50\% y 30\%.\\
Para el conjunto de datos \emph{adult} se han entrenado las distintas herramientas con el 80\%, 50\% y 30\%.\\

{\Huge \textbf{Meter datos aqui}}\\

En cuanto a los problemas que han surgido, con el \emph{dataset} \emph{adult} se necesitaba una columna \emph{class} y por lo tanto, se ha renombrado la última columna a \emph{``Class''}. También otro problema surgido con este \emph{dataset} se ha tenido que convertir la columna \emph{``Class''} de string a número (con el perceptrón multicapa) convirtiendo la cadena \emph{$<=$50K} a la \emph{clase 0} y la cadena \emph{$>$50K} a la \emph{clase 1}. Para ello se ha utilizado un nodo \emph{Column Filter} para eliminar todas las variables no numéricas y un nodo \emph{Category To Number} para convertir la cadena \emph{``Class''} en un entero.\\


{\Huge \textbf{Meter figura workflow 1 aqui}}

{\Huge \textbf{Meter figura workflow 2 aqui}}

\end{document}